\NeedsTeXFormat{LaTeX2e}[2005/12/01]
%%    2009/03/12 v1.0 GAUBM Vorlage f�r Aschlussarbeiten Physik
%% Template fuer Bachelor- und Masterarbeiten
%% an der Fakultaet fuer Physik (c) Thomas Pruschke der GA Universit�t
%% Verbesserungsvorschlaege bitte an pruschke@theorie.physik.uni-goettingen.de
%%
%% Benoetigte Pakete: datenumber
%%

%%%%%%%%%%%%%%%%%%%%%%%%%%%%%%%%%%%%%%%%%%%%%%%%%%%%%%%%%%%%%%%%%%%%%%
%%%%%%%%%% Bitte vor dem Veraendern diese Datei umbenennen! %%%%%%%%%%
%%%%%%%%%%%%%%%%%%%%%%%%%%%%%%%%%%%%%%%%%%%%%%%%%%%%%%%%%%%%%%%%%%%%%%

%% scrbook - Ersatz f�r LaTeX book Klasse aus dem KOMA Script
%% Moegliche Optionen: diejenigen der Klasse scrbook ausser titlepage

%% deutsche Arbeit:
\documentclass[bachelor,       %% Typ der Arbeit: bachelor oder master
               twoside,        %% zweiseitiges Layout
               BCOR10mm,       %% Bindekorrektur 10 mm
%               liststotoc,nomtotoc,bibtotoc, %% Aufnahme der div. Verzeichnisse
                                              %% ins Inhaltsverzeichnis
               english,ngerman, %% Alternativspr. Englisch, Dokumentspr. Deutsch
%               ngerman,english  %% Alternativspr. Deutsch, Dokumentspr. Englisch
%               final,          %% Endversion; draft fuer schnelles Kompilieren
               ]{GAUBM}

\usepackage{setspace}  %% Zur Setzung des Zeilenabstandes
\usepackage{babel}     %% Sprachen-Unterstuetzung
\usepackage{calc}      %% ermoeglicht Rechnen mit Laengen und Zaehlern
\usepackage[T1]{fontenc}       %% Unterstutzung von Umlauten etc.
\usepackage[utf8x]{inputenc}  %%
%% in aktuellem Linux & MacOS X wird standardmaessig UTF8 kodiert!
%\usepackage[utf8]{inputenc}    %% Wenn latin1 nicht geht ...

\usepackage{amsmath,amssymb} %% zusaetzliche Mathe-Symbole

\usepackage{lmodern} %% type1-taugliche CM-Schrift als Variante zur
                     %% "normalen" EC-Schrift
%% Paket fuer bibtex-Datenbanken
\usepackage[comma,numbers,sort&compress]{natbib}
\bibliographystyle{plainnat}

\newcommand{\tabheadfont}[1]{\textbf{#1}} %% Tabellenkopf in Fett
\usepackage{booktabs}                      %% Befehle fuer besseres Tabellenlayout
\usepackage{longtable}                     %% umbrechbare Tabellen
\usepackage{array}                         %% zusaetzliche Spaltenoptionen

%% umfangreiche Pakete fuer Symbole wie \micro, \ohm, \degree, \celsius etc.
\usepackage{textcomp,gensymb}

%\usepackage{SIunits} %% Korrektes Setzen von Einheiten
\usepackage{units}   %% Variante fuer Einheiten

%% Hyperlinks im Dokument; muss als eines der letzten Pakete geladen werden
\usepackage[pdfstartview=FitH,      % Oeffnen mit fit width
            breaklinks=true,        % Umbrueche in Links, nur bei pdflatex default
            bookmarksopen=true,     % aufgeklappte Bookmarks
            bookmarksnumbered=true  % Kapitelnummerierung in bookmarks
            ]{hyperref}

%% Weiter benoetigte Pakete: datenumber
%% Falls dieses Paket nicht in der Installation vorhanden ist,
%% kann es von der Seite mit diesem Template heruntergeladen werden
%% und in einem LaTeX bekanntem Verzeichnis installiert werden (notfalls
%% dem Verzeichnis mit der Arbeit).
\begin{document}
%%
%%                   Ab hier muessen die Anpassungen geschehen
%%
%% Hier den eigenen Namen einsetzen
\ThesisAuthor{William}{Bode}
%% Hier den Geburtsort einsetzen
\PlaceOfBirth{Helmstedt}
%% Titel Arbeit. Das erste Argument ist der deutsche, das zweite der
%% englische Titel.
\ThesisTitle{Molecular Crowding}{}
%% Erst- und Zweitgutacher/in
%% Ist der/die Betreuer/in nicht identisch mit dem/r Erstgutachter/in,
%% muss diese/r als optionales Argument angegeben werden.
\FirstReferee[Dr.\ \ldots]{Prof.\ Dr.\ Stefan Klumpp}
\Institute{Institut für nichtlineare Dynamik / Theoretische Biophysik}
%\SecondReferee{Prof.\ Dr.\ \dots}
%% Beginn und Ende des Anfertigungszeitraumes
\ThesisBegin{1}{2}{2016}
\ThesisEnd{1}{3}{2016}
%% DO NOT TOUCH THESE LINES!!!!
\frontmatter
\maketitle
\cleardoublepage
%% Zusammenfassung. Falls nicht gewuenscht, bitte auskommentieren.
%%\begin{abstract}
%%  Hier werden auf einer halben Seite die Kernaussagen der Arbeit
%%  zusammengefasst.
%% Optional: Stichwoerter. Wenn nicht gewuenscht, koennen die beiden
%% folgenden Zeilen geloescht werden
%%  \bigskip\par
%%  \textbf{Stichwörter:} Biophysik, Molecular Crowding
%%\end{abstract}
%% So laesst sich in die andere Sprache umschalten (Englisch bzw. Deutsch)
%%\begin{otherlanguage}{english}
%%\begin{abstract}
%%  Here the key results of the thesis can be presented in about
%%  half a page.
%%  \bigskip\par
%%  \textbf{Keywords:} Physics, Bachelor thesis
%%\end{abstract}
%%\end{otherlanguage}

%% Ende des Vorspanns
\cleardoublepage
%% Ab hier 1 1/2 facher Zeilenabstand (durch setspace-Paket)
\onehalfspacing
%% Erzeugt Inhaltsverzeichnis
\tableofcontents

%% Hier kann man seine Bezeichnungsweisen erklaeren. Falls nicht
%% benoetigt, bis einschliesslich \end{nomenclature} auskommentieren
\begin{nomenclature}
%% Fuer die Berechnung der Spaltenbreiten muss \usepackage{calc}
%% geladen sein!
\section*{Lateinische Buchstaben}
\noindent
\begin{longtable}[l]{p{0.2\textwidth}p{0.7\textwidth-6\tabcolsep}p{0.1\textwidth}}
  \tabheadfont{Variable}&\tabheadfont{Bedeutung}&\tabheadfont{Einheit}\\\midrule\endhead
  $A$ & Querschnittsfl"ache & $\unit{m^2}$\\
  $c$ & Geschwindigkeit & $\unitfrac{m}{s}$
\end{longtable}
\section*{Griechische Buchstaben}
\begin{longtable}[l]{p{0.2\textwidth}p{0.7\textwidth-6\tabcolsep}p{0.1\textwidth}}
  \tabheadfont{Variable}&\tabheadfont{Bedeutung}&\tabheadfont{Einheit}\\\midrule\endhead
  $\alpha$  & Winkel & $\unit{\degree}$; --\\
  $\varrho$ & Dichte & $\unitfrac{kg}{m^3}$
\end{longtable}
\section*{Indizes}
\begin{longtable}[l]{p{0.2\textwidth}p{0.8\textwidth-4\tabcolsep}}
  \tabheadfont{Index}&\tabheadfont{Bedeutung}\\\midrule\endhead
  m & Meridian\\
  $r$ & Radial
\end{longtable}
\section*{Abk"urzungen}
\begin{longtable}[l]{p{0.2\textwidth}p{0.8\textwidth-4\tabcolsep}}
  \tabheadfont{Abk"urzung}&\tabheadfont{Bedeutung}\\\midrule\endhead
  2D & zweidimensional\\
  3D & dreidimensional\\
  max & maximal
\end{longtable}
\end{nomenclature}
%% \listoftables und \listoffigures sollten nur bei genuegender Anzahl Tabellen
%% verwendet werden
%\listoffigures
%\listoftables

\mainmatter   %% Anfang Hauptteil

\chapter{Motivation}
Das Innere einer Zelle ist so eng besetzt, dass die mittleren Abstände zwischen
benachbarten Proteinen vergleichbar mit der Größe der Proteine selbst ist.
Daher sind Betrachtungen, die von einer freien Bewegung in einer Zelle ausgehen,
nicht ausreichend für eine korrekte Beschreibung von Diffusionskinetik und Gleichgewichtszuständen.
Daher soll in diesem Spezialisierungspraktikum der Einfluss von \emph{Crowding}
untersucht werden, also dem dichten Besetzen einer Zelle mit Proteinen, die weder
Ligand noch Rezeptor in einer zu betrachtenden Reaktion sind.


\chapter{Grundlagen}
\section{Betrachtung mittels Statistischer Mechanik}
Ein wichtiges Werkzeug zur Betrachtung einer Zelle ist der Formalismus der
Statistischen Mechanik. Die Zelle kann als chemisches oder mechanisches Gleichgewicht
betrachtet werden, in der die niedrige Energiezustände und Entropieerhöhung angestrebt
werden. Das Ziel soll nun sein, eine Boltzmannverteilung herzuleiten, welche die
Wahrscheinlichkeit eines Mikrozustands basierend auf einer Energie beschreibt.
Zunächst wird die Boltzmannverteilung nur für Liganden und Rezeptor hergeleitet,
dann um Crowding erweitert
Die folgenden Berechnungen sind im wesentlichen Kapitel 6 und 14 von
\cite{phybio} nachvollzogen.

\subsection{Partikelarrangements Zählen}

Betrachtet man ein Gitter mit $\Omega$ Plätzen auf dem $L$ Liganden zu verteilen
sind, so gibt es für den ersten Liganden $\Omega$ mögliche Plätze. Danach gibt es
für den nächsten der $L - 1$ möglichen Plätze $\Omega - 1$ mögliche Plätze.
Dieses Argument wiederholt man insgesamt $L$-mal und kommt auf die möglichen Arrangements:

\begin{equation}
\text{Anzahl Arrangements} = \frac{\Omega!}{(\Omega\text{-}L)!}
\end{equation}

In unserem Fall sind die Liganden aber ununterscheidbar. Wenn also 2 Liganden
Plätze tauschen würden, so würde sich daraus kein anderer Zustand ergeben. Um
dieses Überzählen zu verhindern, führt man den \emph{Gibbs-Faktor} $\frac{1}{L!}$
ein:

\begin{equation}
\label{zustaende}
\text{Anzahl Zustände} = \frac{\Omega!}{L!(\Omega\text{-}L)!}
\end{equation}

\subsection{Wahrscheinlichkeit von Mikrozuständen}
Das Grundsätzliche Ziel der statistischen Mechanik ist es, Wahrscheinlichkeitsaussagen
über das Auftreten verschiedener Mikrozustände zu treffen. Der wesentliche Unterschied
zwischen zwei Mikrozuständen ist die Energie $E_i$, wobei $i$ den $i$-ten Mikrozustand
bezeichnet. Aus der Statistischen Mechanik ist nun genau die Wahrscheinlichkeit
bekannt, mit der jeweils ein Mikrozustand auftritt:
\begin{equation}
  \label{eq:boltzmann}
  p(E_i) = \frac{1}{Z} e^{\text{-}E_i\text{/}k_bT}
\end{equation}

Der Faktor $\frac{1}{Z}$ dient der Normierung der Verteilung.
Z ist bekannt als die Zustandssumme. Die Gleichung \ref{eq:boltzmann}
ist bekannt als Boltzmann-Verteilung. Die Normalisierungskonstante $\frac{1}{Z}$
kann durch die Anforderung $ \sum_{i=1}^{N} p(E_i) = 1$ bestimmt werden, was bedeutet:
\begin{equation}
Z = \sum_{i=1}^{N}e^{\text{-}E_i\beta}
\end{equation}

$\beta$ ist hier die inverse Temperatur $\frac{1}{k_bT}$

\subsection{Bindewahrscheinlichkeit}
Diese Werkzeuge können nun angewandt werden, um die Bindewahrscheinlichkeit eines
Liganden an einem Rezeptor zu berechnen.

Hierzu bilden wir das Verhältnis aus dem statistischen Gewicht des gebundenen
Zustands und der Summe der statistischen Gewichte der gebundenen und ungebundenen
Zustände. Das Gewicht des gebundenen Zustands ist
$$e^{\text{-}\beta\varepsilon_b} \cdot \sum_{Lösung}e^{\text{-}\beta(L\text{-}1)\varepsilon_{sol}}$$
Wobei der erste Term den gebundenen Liganden beschreibt und der zweite Term die Summe
über alle ungebundenen Liganden in der Lösung ist. Wir haben hier $\varepsilon_b$ als
die Bindungsenergie und $\varepsilon_{sol}$ als die Energie für ein freies Teilchen
in der Lösung eingeführt. Die Summe über die Lösung ist eine Anweisung dafür,
über alle Konfigurationen von $L-1$ Teilchen auf $\Omega$ Plätzen zu summieren,
wobei jede davon das Gewicht $e^{-\beta(L-1)\varepsilon_{sol}}$ hat.
Da der Boltzmannfaktor für all diese Zustände gleich ist, gilt es nur
die Anzahl an Konfigurationen zu finden, das führt zu:

\begin{equation}
\label{gebundenerligand}
\sum_{Lösung}e^{\text{-}\beta(L\text{-}1)\varepsilon_{sol}} = \frac{\Omega!}{(L-1)![\Omega-(L-1)]!}e^{-\beta(L-1)\varepsilon_{sol}}
\end{equation}

Desweiteren benötigen wir für die weitere Betrachtung die Zustandssumme,
welche eine Summe ist aus den möglichen gebundenen und ungebundenen Zuständen, also:

\begin{equation}
Z(L,\Omega) = \sum_{Lösung}e^{-\beta L\varepsilon_{sol}} + e^{-\beta\varepsilon_b}\sum_{Lösung}e^{-b(L-1)\varepsilon_{sol}}
\end{equation}

Wir haben den zweiten Term schon in \ref{gebundenerligand} bestimmt, daher fehlt noch der erste.
Hierzu kommen wir mit Gleichung \ref{zustaende} auf:

\begin{equation}
\sum_{Lösung}e^{-\beta L\varepsilon_{sol}} = e^{-\beta L\varepsilon_{sol}}\frac{\Omega!}{L!(\Omega\text{-}L)!}
\end{equation}

Somit können wir die Zustandssumme wie folgt ausdrücken:
\begin{equation}
Z(L,\Omega) = e^{-\beta L\varepsilon_{sol}}\frac{\Omega!}{L!(\Omega\text{-}L)}! + e^{-\beta\varepsilon_b}e^{-\beta(L-1)\varepsilon_{sol}} \frac{\Omega!}{(L-1)![\Omega-(L-1)]!}
\end{equation}

Man kann nun folgende Näherung rechtfertigen:\cite{phybio}
\begin{equation}
\frac{\Omega!}{(\Omega-L)!} \approx \Omega^L
\end{equation}

Nun können wir die Bindewahrscheinlichkeit $p_{bound}$ folgendermaßen ausdrücken:
\begin{equation}
p_{bound} = \frac{e^{-\beta\varepsilon_b}\frac{\Omega^{L-1}}{(L-1)!}e^{-\beta(L-1)\varepsilon_{sol}}}{\frac{\Omega^L}{L!}e^{-\beta L\varepsilon_{sol}} + e^{-\beta\varepsilon_b}\frac{\Omega^{L-1}}{(L-1)!}e^{-\beta(L-1)\varepsilon_b}}
\end{equation}
erweitern wir diesen Bruch um $(L!/\Omega^L)e^{\beta L\varepsilon_{sol}}$ und definieren
$\delta\varepsilon = \varepsilon_b -\varepsilon_{sol}, so vereinfacht sich der Ausdruck zu:

\begin{equation}
\end{equation} 

Abbildung~\ref{fig:bildplatzhalter} verdeutlicht \dots

Wie die Abb.~\ref{fig:bildplatzhalter} und
Tab.~\ref{tab:tabellenplatzhalter} verdeutlichen \dots

\begin{figure}
  \centering
  \includegraphics[width=0.5\linewidth]{figures/bild}
  \caption{Bildbeschreibung}
  \label{fig:bildplatzhalter}
\end{figure}

Text\dots
\begin{table}
  \centering
  \begin{tabular}{llll}
    \toprule
    $A$-Wert&$B$-Wert&$C$-Wert&$D$-Wert\\
    \midrule
    aaaaaa&bbbbbbb&cccccc&ddddddd\\
    aaaaaa&bbbbbbb&cccccc&ddddddd\\
    \bottomrule
  \end{tabular}
  \caption{Tabellenbeschreibung}
  \label{tab:tabellenplatzhalter}
\end{table}


%%\appendix
%%\chapter{erster Anhang}
%%Text\dots
%%\chapter{zweiter Anhang}
%%Text\dots

\cleardoublepage
%% Bibliographie. Das Argument muss der Name der BIBTeX-Datenbank stehen.
%% Ein Beispiel fuer eine solche Datenbank finden Sie in bthesis_datenbank.bib
\bibliography{bthesis_datenbank}

%%\chapter*{Danksagung}
%%Dank\dots

%% Dieser Befehl MUSS am Ende stehen und erzeugt die Erklaerung ueber die
%% benutzten Mittel
%%\Declaration
\end{document}
